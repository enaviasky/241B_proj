
\documentclass[conference]{IEEEtran}
\usepackage{blindtext, graphicx}
\ifCLASSINFOpdf
\else
 \fi
\hyphenation{op-tical net-works semi-conduc-tor}

\begin{document}

\title{Energy and Power Efficient Convolutional Neural Network with Approximate Computing}

\author{\IEEEauthorblockN{Keertana Settaluri}
\IEEEauthorblockA{School of Electrical and Computer Engineering\\
University of California, Berkeley\\
Email: ksettaluri6@berkeley.edu}
\and
\IEEEauthorblockN{Emily Naviasky}
\IEEEauthorblockA{School of Electrical and Computer Engineering\\
University of California, Berkeley\\
Email: enaviasky@berkeley.edu}
}

\maketitle

\begin{abstract}
%\boldmath
\blindtext[1]
\end{abstract}

\begin{IEEEkeywords}
IEEEtran, journal, \LaTeX, paper, template.
\end{IEEEkeywords}

\IEEEpeerreviewmaketitle

\section{Introduction}
	Neural networks have gained immense popularity in recent decades, primarily because of their utility in applications such as computer vision, speech recognition, and natural language processing. Creating a hardware implementation of a neural network, however, is extremely difficult due to the sheer amount of computation required. For example, AlexNet [1] uses 2.3 million weights, and requires 666 million MACs per 227x227 image. VGG16 [2] uses 14.7 million weights, and requires a staggering 15.3 billion MACs per 224x224 image. Understandably, a significant amount of time and research is being used to develop a more efficient, less power intensive, and faster hardware implementation of a neural network.\\
	\indent Although low power implementations have been developed, they are usually at the expense of generalization, wherein specialized dataflow and hardware are developed to minimize data movement and skip unnecessary read or write operations, as in the case of Eyeriss [3]. This hardware specific optimization limits the application to only the one in which it was designed. 

\subsection{Approximate Computing}
	Hardware specific approximate computing is an optimization approach that efficiently reduces the precision of multiply and accumulate operations in order to speed up a system and save power. \\
	\indent Many approximate adders and multipliers have been designed to reduce energy at the expense of precision. The Approximate Mirror Adder (AMA), for example, removes transistors at the logic level, thus reducing node capacitance while inducing error in the output, and predicts up to 69\% [4] savings in power when four out of 16 bits are implemented using an AMA. Voltage Overscaling (VOS) is another design technique that reduces the supply voltage 
	
	
% needed in second column of first page if using \IEEEpubid
%\IEEEpubidadjcol

% An example of a floating figure using the graphicx package.
% Note that \label must occur AFTER (or within) \caption.
% For figures, \caption should occur after the \includegraphics.
% Note that IEEEtran v1.7 and later has special internal code that
% is designed to preserve the operation of \label within \caption
% even when the captionsoff option is in effect. However, because
% of issues like this, it may be the safest practice to put all your
% \label just after \caption rather than within \caption{}.
%
% Reminder: the "draftcls" or "draftclsnofoot", not "draft", class
% option should be used if it is desired that the figures are to be
% displayed while in draft mode.
%
%\begin{figure}[!t]
%\centering
%\includegraphics[width=2.5in]{myfigure}
% where an .eps filename suffix will be assumed under latex, 
% and a .pdf suffix will be assumed for pdflatex; or what has been declared
% via \DeclareGraphicsExtensions.
%\caption{Simulation Results}
%\label{fig_sim}
%\end{figure}

% Note that IEEE typically puts floats only at the top, even when this
% results in a large percentage of a column being occupied by floats.


% An example of a double column floating figure using two subfigures.
% (The subfig.sty package must be loaded for this to work.)
% The subfigure \label commands are set within each subfloat command, the
% \label for the overall figure must come after \caption.
% \hfil must be used as a separator to get equal spacing.
% The subfigure.sty package works much the same way, except \subfigure is
% used instead of \subfloat.
%
%\begin{figure*}[!t]
%\centerline{\subfloat[Case I]\includegraphics[width=2.5in]{subfigcase1}%
%\label{fig_first_case}}
%\hfil
%\subfloat[Case II]{\includegraphics[width=2.5in]{subfigcase2}%
%\label{fig_second_case}}}
%\caption{Simulation results}
%\label{fig_sim}
%\end{figure*}
%
% Note that often IEEE papers with subfigures do not employ subfigure
% captions (using the optional argument to \subfloat), but instead will
% reference/describe all of them (a), (b), etc., within the main caption.


% An example of a floating table. Note that, for IEEE style tables, the 
% \caption command should come BEFORE the table. Table text will default to
% \footnotesize as IEEE normally uses this smaller font for tables.
% The \label must come after \caption as always.
%
%\begin{table}[!t]
%% increase table row spacing, adjust to taste
%\renewcommand{\arraystretch}{1.3}
% if using array.sty, it might be a good idea to tweak the value of
% \extrarowheight as needed to properly center the text within the cells
%\caption{An Example of a Table}
%\label{table_example}
%\centering
%% Some packages, such as MDW tools, offer better commands for making tables
%% than the plain LaTeX2e tabular which is used here.
%\begin{tabular}{|c||c|}
%\hline
%One & Two\\
%\hline
%Three & Four\\
%\hline
%\end{tabular}
%\end{table}


% Note that IEEE does not put floats in the very first column - or typically
% anywhere on the first page for that matter. Also, in-text middle ("here")
% positioning is not used. Most IEEE journals use top floats exclusively.
% Note that, LaTeX2e, unlike IEEE journals, places footnotes above bottom
% floats. This can be corrected via the \fnbelowfloat command of the
% stfloats package.

\section{Problem Description}
-is it worth doing approximate computing in CNN?
CNN are a error tolerant application that is prevalent in many power conscious applications even though it requires a large amount of computation to train and use. It is an easy decision to want to trade some of that unnecessary accuracy for power savings and faster computation. However, accuracy for speed and power is not a decision made in a vacuum. Area on silicon is expensive and an entire block for approximate hardware would have to have significant enough power savings to be worth the area. <Was there more reasons besides area? I can't remember> It is the goal of this paper to explore whether CNN are an appropriate application for approximate computing in a quantitative way. We will begin <fuck not using I or we, fix it later> by examining CNNs and approximate computing in more detail so that we can <talk intelligently about them later.>
\subsection{Convolutional Neural Nets}
-What is a CNN? 
CNNs address the problem common to using neural nets in computer vision, which is that it requires an infeasible number of neural nodes<?> in order to process a very small image if each pixel is given it's own node. <lots of nn stuff to ask keertana>
	-CNNs are for vision
	-how they differ from normal NNs (course thing/papers on CNNs)
-algorithms and noise that are typically in the system
-FOM, how their goodness is judged(error in NNs)
-What is approximate computing?
There are many implementations of approximate adders and multipliers. 
	-Describe approx comp we want to use
		-this one is superior cuz BLAH
	-what are some typical numbers for improvement
		-they saw X power/ speed improvement vs error
	-how we are going to model and judge FOM
-paper on approximate CNN
-They look like a good fit + decent ending sentence

\section{Solution}
-Why should approximate computing in CNNs work?
	-quote some power numbers or something
-judge workspace
		-other works for X loss in accuracy see gain in power/speed
		-graph power vs accuracy/ accuracy necessary
			-how much accuracy does a NN need anyway?
-paper on approximate CNN
	-more depth, depending on how useful it was
-What we hope to find
	-Power and speed improvement is significant versus loss of accuracy
	-Reiterate main question: is it worth doing approximate computing in CNN?

\section{Description of Experimental Work}
-How are we going to judge worth or not worth
-Experimental setup
	-verilog for power numbers, size, error rate
-python, lanet (reference more course thing), blah blah don’t wanna train one
	-plug in error rate/introduce same amount of error
-see if shit still works
	- judge goodness
-FOM
	-what are we doing in particular to judge power/speed/usability of NN


\section{Conclusion}
-WOOOO HOPE IT WORKS




% if have a single appendix:
%\appendix[Proof of the Zonklar Equations]
% or
%\appendix  % for no appendix heading
% do not use \section anymore after \appendix, only \section*
% is possibly needed

% use appendices with more than one appendix
% then use \section to start each appendix
% you must declare a \section before using any
% \subsection or using \label (\appendices by itself
% starts a section numbered zero.)
%


\appendices
\section{Proof of the First Zonklar Equation}
\blindtext

% use section* for acknowledgement
\section*{Acknowledgment}


The authors would like to thank...


% Can use something like this to put references on a page
% by themselves when using endfloat and the captionsoff option.
\ifCLASSOPTIONcaptionsoff
  \newpage
\fi



% trigger a \newpage just before the given reference
% number - used to balance the columns on the last page
% adjust value as needed - may need to be readjusted if
% the document is modified later
%\IEEEtriggeratref{8}
% The "triggered" command can be changed if desired:
%\IEEEtriggercmd{\enlargethispage{-5in}}

% references section

% can use a bibliography generated by BibTeX as a .bbl file
% BibTeX documentation can be easily obtained at:
% http://www.ctan.org/tex-archive/biblio/bibtex/contrib/doc/
% The IEEEtran BibTeX style support page is at:
% http://www.michaelshell.org/tex/ieeetran/bibtex/
%\bibliographystyle{IEEEtran}
% argument is your BibTeX string definitions and bibliography database(s)
%\bibliography{IEEEabrv,../bib/paper}
%
% <OR> manually copy in the resultant .bbl file
% set second argument of \begin to the number of references
% (used to reserve space for the reference number labels box)
\begin{thebibliography}{1}

\bibitem{IEEEhowto:kopka}
H.~Kopka and P.~W. Daly, \emph{A Guide to \LaTeX}, 3rd~ed.\hskip 1em plus
  0.5em minus 0.4em\relax Harlow, England: Addison-Wesley, 1999.

\end{thebibliography}

% biography section
% 
% If you have an EPS/PDF photo (graphicx package needed) extra braces are
% needed around the contents of the optional argument to biography to prevent
% the LaTeX parser from getting confused when it sees the complicated
% \includegraphics command within an optional argument. (You could create
% your own custom macro containing the \includegraphics command to make things
% simpler here.)
%\begin{biography}[{\includegraphics[width=1in,height=1.25in,clip,keepaspectratio]{mshell}}]{Michael Shell}
% or if you just want to reserve a space for a photo:

\begin{IEEEbiography}[{\includegraphics[width=1in,height=1.25in,clip,keepaspectratio]{picture}}]{John Doe}
\blindtext
\end{IEEEbiography}

% You can push biographies down or up by placing
% a \vfill before or after them. The appropriate
% use of \vfill depends on what kind of text is
% on the last page and whether or not the columns
% are being equalized.

%\vfill

% Can be used to pull up biographies so that the bottom of the last one
% is flush with the other column.
%\enlargethispage{-5in}




% that's all folks
\end{document}


